\documentclass{article}
\usepackage[utf8]{inputenc}
\usepackage[T1]{fontenc}
\usepackage{setspace}



\title{Tasks}
\author{Phil Reinartz}
\date{00.00.0000}

\begin{document}

\maketitle

\section{Task1)}

\textbf{Answer 1.2} The main function of the preprocessor is, that it adds the "full" c code into your .i data. For example, the stdio.h librarie is nothing other than some fuctions that someone declared, so that you don't have to write them yourself. These functions are loaded by the preprocessor in your programm, because you wrote \#include <stdio.h> inside your code. The same is also for makros that you have placed. For example: if you wrote "\#define Maximalwert 1234567890987654321234567899999999", the preprocessor takes the value of Maximalwert and puts it everywhere in the programm where you wrote Maximalwert. So all in all, the preprocessor does nothing other, than complete your programm by replacing predefined makros with the right value and adding and suptracting content you want to include, or your machine needs to work right.
So that your programm is ready to be compiled in the next step.




\section{1.3)}

\textbf{Answer:} Hpjhnefoisjhblsjhnbvosjhbvls,jhb
\end{document}
